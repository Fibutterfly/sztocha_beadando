\documentclass[11pt,a4pape,draftr]{article}
	\usepackage[iso]{datetime}
	\usepackage[hungarian]{babel}
	\usepackage{amssymb}
	\usepackage{amsmath}

	\title{Nemlineáris egyenletrendszerek megoldása}
	\date{\today}
	\author{Filep Illés Attila}

\begin{document}
 	\maketitle
	\pagenumbering{gobble}
  	\newpage
  	\pagenumbering{arabic}

	\begin{abstract}

	\end{abstract}
	\tableofcontents
	\section*{Bevezetés}
		\paragraph{}
			A dokumentum célja a sztochasztikus folyamatok alkalmazása nevű tárgyon tanult, kiemelt elemek demonstrációja. A demonstráció MATLAB könyvtár elkészítésével történik. A könyvtárnak a célja, hogy szimbolikus matematikai eszközökkel a folyamatokat bemutassa. A könyvtárnak nem célja a semmilyen informatikai optimalizáltságot megvalósítani.
	\part{Stacionárius folyamatok}
		\section{Alapvető definíciók}
			\subsection{Valószínűségi változó}
				Legyen:
				\begin{itemize}
					\item $\Omega$ egy nem üres halmaz
					\item $\{\omega : X(\omega) < x\} \in \mathcal{A}$
					\item $x \in \Bbb{R}$
					\item $\mathcal{A}$ az $\Omega$ részhalmazaiból alkotott esemény $\sigma$-algebrája (tehát $(\Omega, \mathcal{A})$ mérhető tér)
				\end{itemize}
				Akkor $X: \Omega \to \Bbb{R}$ függvényt valószínűségi változónak hívunk.
			\subsection{Sztochasztikus folyamat}
				\paragraph{}
					A sztochasztikus folyamat (vagy véletlen folyamat) egy olyan matematikai modell, amely egy vagy több időfüggő véletlen változó által létrehozott folyamatot ír le. A sztochasztikus folyamatok olyan rendszerek leírására szolgálnak, amelyekben a jövő állapota részben véletlenszerűen határozza meg a múlt és a jelen állapotát.
				\paragraph{}
					A sztochasztikus folyamatok általában valószínűségi változók sorozataként jelennek meg, amelyeknek az idő függvényében változó értékei vannak. A folyamatot gyakran matematikailag leírt egyenletekkel vagy valószínűségi eloszlásokkal írják le.
				\paragraph{}
					A sztochasztikus folyamatok számos területen alkalmazhatók, például az anyag- és energiaátvitel, a kommunikációs rendszerek, a pénzügyek, az idősorok elemzése és a gépi tanulás területén.
				\subsubsection{Sztochasztikus folyamatok kompatibilitási feltételei}
					A sztochasztikus folyamatok kompatibilitási feltételei a következők:
					\begin{itemize}
					   \item Az időpillanatok száma felsorolható, véges vagy végtelen, de számontartható.
					   \item Az időpillanatok sorozata szigorúan növekvő, azaz $t_1 < t_2 < \cdots < t_n$ vagy $t_1 < t_2 < \cdots < t_{\infty}$.
					   \item Az időpillanatok közötti időközök meghatározottak és végesek vagy végtelenek.
					   \item A folyamat értékei véletlenszerűek, és általában valószínűségi változóként vannak definiálva.
					   \item A folyamat értékei időfüggők, és az időbeli elmozdulásokkal szembeni szimmetriára vonatkozó korlátozásokat kell teljesítenie. Például a stacionárius folyamatok esetében az eloszlások nem változnak az idő múlásával, és az átlag és szórás időfüggetlen.
					\end{itemize}
					Ezen kívül a sztochasztikus folyamatoknál általában szükséges az ergodicitás feltétele, amely azt jelenti, hogy a folyamat minden pillanatban eléri minden lehetséges állapotát az idő végtelen futamán. Ez fontos feltétele a statisztikai tulajdonságok meghatározásának, mert lehetővé teszi a folyamat várható értékének becslését a mintavételezés révén.
			\subsection{Várható érték}
				\paragraph{}
					Lényegében az első (centrális) momentum, egy funkciónál.
				\paragraph{Diszkrét esetben}
					$$E(X) = \sum_{i=1}^{\infty}p_i x_i$$
				\paragraph{Folytonos esetben}
					$$E(X) = \int_{-\infty}^{\infty}x f(x)dx$$
				\subsubsection{Várható érték létezésének a feltétele}
					\paragraph{Diszkrét esetben}
						$$E(X) = \sum_{i=1}^{\infty}p_i |x_i| < \infty$$
					\paragraph{Folytonos esetben}
						$$E(X) = \int_{-\infty}^{\infty}|x| f(x)dx < \infty$$
	
				\subsubsection{Várható érték tulajdonságai}
					\begin{itemize}
						\item Ha $X$ az $1$ valószínűséggel korlátos valószínűségi változó, akkor van olyan $x_1$ és $x_2$ konstans, hogy $P(x_1 \le X \le x_2)=1$ akkor $x_1 \le E(X) \le x_2$
						\item $E(cX)=cE(X)$
						\item $P(X=c) = 1 \to E(X)=c$
						\item $E(X+Y) = E(X) + E(Y)$
						\item $E(X*Y)=E(X)*E(Y)$
					\end{itemize}
	
			\subsection{Kovariancia függvény}
				$$ \begin{aligned}
					R_X(u) &= \operatorname{cov}(X_t, X_{t-u}) \\
					&= E[(X_t - E(X_t))(X_{t-u}-E(X_{t-u}))]
				\end{aligned} $$
	
				Ez itt nem a kovarianciamátrixot fogja vissza adni, hanem az eltérés közötti összefüggést.
				\subsubsection{Kovariancia függvény tulajdonságai}
					\begin{itemize}
					\item Additivitás: Ha $X$ és $Y$ véletlen változók és $a$ és $b$ valós számok, akkor a kovarianciafüggvény additív, azaz:
					$$cov(aX + bY, Z) = a_cov(X, Z) + b_cov(Y, Z)$$
					
					\item Szimmetria: A kovarianciafüggvény szimmetrikus, azaz
					$$cov(X,Y) = cov(Y,X)$$
					
					\item Állandóság: Ha $X$ és $Y$ véletlen változók és $a$ és $b$ konstansok, akkor a kovarianciafüggvény állandó marad, ha mindkét változót $a$-val és $b$-vel eltoljuk. Azaz,
					$$cov(X + a, Y + b) = cov(X, Y)$$
					
					\item Nemnegativitás: A kovarianciafüggvény mindig nemnegatív, azaz
					
					$$cov(X, X) \ge 0$$
					
					Ha a két változó független, akkor az egyenlőség akkor és csak akkor áll fenn, ha az $X$ állandó.
					
					\item Normálás: Ha $X$ és $Y$ normális eloszlásúak, akkor a kovarianciafüggvény teljesen meghatározza a két változó közötti kapcsolatot.
					
					\item Két független változó kovarianciája nulla: Ha $X$ és $Y$ független változók, akkor a kovarianciafüggvényük zérus:
					
					$$cov(X, Y) = 0$$
					\end{itemize}
			\subsection{Gauss folyamat}
			Egy folyamatot Gauss folyamatnak nevezünk, ha a következő tulajdonságokkal rendelkezik:
			\begin{itemize}
				\item Az összes véges dimenziós eloszlása Gauss-eloszlású kell legyen. Ez azt jelenti, hogy az összes véges dimenziós eloszlásfüggvény szimmetrikus, és a karakterisztikus függvénye exponenciális alakú kell legyen.
				\begin{itemize}
					\item Korreláció mátrixokat mind meg kell nézni, hogy pozitívak-e.
				\end{itemize}
					\item Az összes időpillanatra vonatkozó középérték és szórás azonos kell legyen. A folyamat homogénnek tekinthető.
				\begin{itemize}
					\item Ezt homogenitás teszttel lehet ellenőrizni.
				\end{itemize}
				\item Az összes időpillanatban értékeket vesz fel végtelen dimenziós vektorokban. A végtelen dimenziós eloszlás azonban nem kell Gauss-eloszlásúnak lennie.
			\end{itemize}
			\subsection{Herglotz-tétel}
				\paragraph{}
					Legyen $R_X (u)$ a folyamat kovarianciafüggvénye, és tegyük fel, hogy ez a függvény az időbeli eltolásra invariáns, azaz csak a két időpont közötti különbségtől függ. Ekkor $R_X (u)$ Herglotz-féle sűrűségfüggvényként is felírható, azaz teljesül rá a következő:

					$$R_X (u)=\int_{-\infty}^\infty e^{i*\lambda*u}*g_X(\lambda)d\lambda$$

					ahol $g_X(\lambda)$ egy valós, szigorúan monoton növekvő eloszlásfüggvény. Más szóval, a kovarianciafüggvény Fourier-transzformáltját egy valós eloszlásfüggvénnyel lehet leírni.
			\subsection{Trajektória}
				\paragraph{}
					Egy tetszőleges $X_t$ folyamat trajektóriái alatt a folyamat lehetséges megvalósulását értjük.
					\begin{itemize}
						\item $X_t$ tetszőleges sztochasztikus folyamat
					\end{itemize}
			\subsection{Nagy Számok Erős Törvénye}
				\paragraph{}
					$$P\left( \lim_{n \to \infty} \dfrac{\sum_{i=1}^n X_i}{n} = c\right) = 1$$
					\begin{itemize}
						\item $c = E(X_1)= E(X_2) = \dots = E(X_n)$
						\item $X_i$ független azonos eloszlású valószínűségi változó
					\end{itemize}.
			\subsection{iterált logaritmus tétel}
				\paragraph{}
					A következő összefüggések $1$ valószínűséggel fennállnak
					$$\lim_{n \to \infty} \sup \dfrac{\sum_{i=1}^n(X_i) - n* E(X_n)}{\sqrt[2]{2*n*\ln(\ln(n))}} = + | \sigma|$$
					és
					$$\lim_{n \to \infty} \inf \dfrac{\sum_{i=1}^n(X_i) - n* E(X_n)}{\sqrt[2]{2*n*\ln(\ln(n))}} = - | \sigma|$$
					Ahol:
					\begin{itemize}
						\item $VAR(X_n) = \sigma^2 < \infty$
						\item $X_i$ független, azonos eloszlású valószínűségi változók
						\item $E(X_n) = E(X_1) = \dots = E(X_{n-1})$
					\end{itemize}.
				\paragraph{}
					A tétel tehát azt mondja ki, hogy az $x$ szám $n$-edik logaritmusát az $x$ szám természetes logaritmusának $n$-edik hatványának az $\frac{1}{\ln(b)}$ faktorral szorozva kapjuk meg.
			\subsection{Stielejtes integrál}
				\paragraph{}
					$$\int_a^b f(x) dg(x) = \lim_{n \to \infty} \sum_{k=1}^n f(x_k)[g(z_k) - g(z_{k1})]$$
					A következő tulajdonságokkal rendelkezik:

					\begin{align*}
						\int_a^b f(x) , dg(x) + \int_a^b g(x) , df(x) &= [f(x) g(x)]_a^b \
						\int_a^b f(x) , dg(x) &= \int_a^b f(x) , g^\prime (x) , dx
					\end{align*}
			\subsection{szimmetriafeltétel}
				\paragraph{}
					Legyen $\{i_1, \dots, i_n\}$ az $1, \dots, n$ számok permutációja, akkor tetszőleges időpontokra és $n \ge 1$-re érvényes, hogy
					$$F_{i_m, \dots, i_n}(x_{i_m}, \dots, x_{i_n}) = F_{i_1, \dots, i_n}(x_{i_1}, \dots, x_{i_n})$$
					ahol:
					\begin{itemize}
						\item $0 < m \le n$ 
					\end{itemize}.
			\subsection{kompatibilitási feltétel}
				\paragraph{}
					$$F_{t_{i_1},\dots t_{i_n}}(x_1, \dots, x_m, y_{m+1}, \dots,y_n) = F_{t_1, \dots,t_m}(x_1, \dots, x_m)$$
					Ahol:
					\begin{itemize}
						\item $0< m < n$
						\item tetszőleges $t_{m+1}, \dots , t_n \in [t_0,T]$
					\end{itemize}
			\subsection{Dirac Delta}
				\paragraph{}
					A Dirac-féle delta függvény egy olyan matematikai objektum, amely egy függvényként viselkedik, de a hagyományos függvényekkel szemben a végtelen sok helyen 0 kivételével 0 értéket vesz fel. Formálisan, ha $f(x)$ egy integrálható függvény, akkor a Dirac-delta függvényt ($\delta(x)$) a következőképpen definiáljuk:
					$$\int_{- \infty}^{\infty}f(x)\delta(x - x_0) dx = f(x_0)$$
					Ez azt jelenti, hogy a Dirac-delta függvény egy olyan függvény, amely az integrál szempontjából viselkedik, mint a végtelenül keskeny, végtelenül magas csúcs, amelynek területe 1. Azonban, a Dirac-delta függvény matematikailag nem egy függvény a hagyományos értelemben, hanem egy úgynevezett "eloszlás". Ez azt jelenti, hogy a Dirac-delta függvény integrálható más függvényekkel, és a fenti egyenlet alapján értelmeztük az értékét egy adott pontban.
		\section{Stacionárius folyamatok}
			\paragraph{}
				A stacionárius folyamatok olyan valószínűségi folyamatok, amelyeknek a statisztikai tulajdonságai nem változnak az idő múlásával. Az ilyen folyamatok esetében a várható érték és a kovariancia függvénye nem függ az időtől, vagyis az idősor jellege nem változik az idő múlásával.
			\paragraph{}
				A stacionárius folyamatok matematikailag jól definiáltak és számos fontos tulajdonsággal rendelkeznek, amelyek lehetővé teszik számunkra az idősorok modellezését és előrejelzését. Az ilyen folyamatokra vonatkozóan meghatározott várható érték és kovariancia függvény jellemzi a folyamatot teljes egészében. 
			\paragraph{}
				A stacionárius folyamatok fontosak a való életben előforduló idősorok modellezésében is, például a gazdasági mutatók és a meteorológiai adatok előrejelzésében. Az ilyen folyamatok matematikai tulajdonságai lehetővé teszik az idősorok előrejelzését, a kockázatbecslést és az optimalizálást. 
			\subsection{Tágabb értelemben stacionárius folyamat}
				\paragraph{}
					Legyen $\{X_t, t \in \mathcal{T}\}$, ahol
					\begin{itemize}
						\item $X$ a $t$ időponthoz tartozó sztochasztikus folyamat
					\end{itemize}
					A tágabb értelemben vett stacionárius folyamatot szokás röviden stacionárius folyamatnak nevezni.
				\paragraph{}
					Ha egy sztochasztikus folyamat tágabb értelemben stacionárius, az azt jelenti, hogy a várható értéke és a kovariancia függvénye csak az időbeli különbségtől függ, és nem az abszolút időponttól.
				\subsubsection{Tesztelés tágabb értelemben vett stacionárius folyamatra}
					\paragraph{}
						Amennyiben a következő feltételek megegyeznek az egy tágabb értelemben vett stacionárius folyamat:
						\begin{itemize}
							\item $\mu_X (t) = E(X_t)$
							\item $R_X(s,t) = cov(X_s, X_t) = E(X_t - \mu_X (t)) (X_s - \mu_X (s))$
						\end{itemize}
						Ahol a következők a következőket jelenti:
						\begin{itemize}
							\item $t, s \in \mathcal{T}$, tehát időbéli változók
							\item $\mu_X (t)$ egy konstans, ami csak az időtől függ és megegyezik a várható értékkel
							\item $R_X$ a kovariancia függvény
							\item $E(X_t^2) < \infty$ 
						\end{itemize}
				\subsubsection{tágabb értelemben stacionárius eloszlásainak feltételei}
					\paragraph{}
						Adott a következő véges dimenziós rendszer:
						$$\begin{aligned}
							P(X_t \le x) &= F_t(x) \\
							P(X_{t_1} \le x_t, X_{t_2} \le x_2) &=F_{t_1,t_2}(x_1,x_2) \\
							\vdots \\
							P(X_{t_1} \le x_1 , \dots, X_{t_n} \le x_n) &= F_{t_1, \dots, t_n}=F(x_1, \dots, x_n)
							\vdots
						\end{aligned}$$
						amely eleget tesz:
						\begin{itemize}
							\item szimmetriafeltételnek
							\item kompatibilitási feltételnek
						\end{itemize}

			\subsection{Szűkebb értelemben stacionárius folyamat}
				\paragraph{}

				Ahhoz, hogy valamit szűkebb értelemben stacionáriusnak nevezzünk teljesülnie kell, hogy $(X_{t_1}, \dots , X_{t_n})$ és $(X_{t_{1+t}}, \dots, X_{t_n + t})$ valószínűségi változók  együttes eloszlása megegyezik és tágabb értelemben stacionárius folyamat.

		\section{Stacionárius Gauss folyamatok}
			\paragraph{}
				A stacionárius Gauss-folyamat olyan Gauss-folyamat, amelynek a statisztikai tulajdonságai (középérték, szórás, autokorrelációs függvény) időtől függetlenek, vagyis az időbeli változások nem befolyásolják ezeket a tulajdonságokat.
		\section{Spektrál előállítás}
			\paragraph{}
				A Herglotz-tétel szerint, a kovarianciafüggvényt kitudjuk fejezni a következő képen:
				$$R_X (u)=\int_{-\infty}^\infty e^{i*\lambda*u}*g_X(\lambda)d\lambda$$
				Ebben az összefüggésben a spektrális sűrűségfüggvény a $g_X(\lambda)$.
			\paragraph{}
				Analóg módon értelmezzük a diszkrét esetet is:
				$$R_X(u) = \sum_{k = - \infty}^\infty \sigma^2_k e^{i*u* \lambda_k}$$
			\subsection{Spektrális sűrűségfüggvény előállítása speciális esetben}
				Ha $\sum_{k=- \infty}^{\infty}|R_X(u)| < \infty$ feltétel teljesül, akkor a spektrális sűrűségfüggvény közvetlenül is előállítható:
				$$g_X(\lambda) = \cfrac{1}{2 \pi} \sum_{k = - \infty} ^{\infty} R_X(k) e^{-i * k *\lambda}$$
				Ez tovább írható:
				$$
				\begin{aligned}
					g_X(\lambda) =& \cfrac{1}{ 2  \pi} \sum_{k = - \infty}^{\infty}R_X(k)cos(k \lambda) = \\
					=& \cfrac{1}{2 \pi} \left( \sigma_X^2 + 2 \sum_{k = 1}^{\infty} R_X(k) cos(k \lambda) \right)
				\end{aligned}$$
			\subsection{Spektrális sűrűségfüggvény segítségével lévő stacionárius folyamat előállítása}
				\paragraph{}
					$$X_t = \mu_X + \int_{- \pi} ^\pi e^{it\lambda} d Z(\lambda)$$
					$$X_t = \mu_X + \sum_{k=-\infty}^{\infty}Z_ke^{i*t*\lambda_k}$$
					\begin{itemize}
						\item $\mu_X$ a várható érték
						\item $t$ az idő változó
						\item $Z(\lambda), -\pi \le \lambda \le \pi$ egy sztochasztikus folyamat, amely zérus várható értékű
						\item $E(Z(\lambda^{\prime\prime} - Z(\lambda^{\prime}))^2 = G_X(\lambda^{\prime \prime}) -G_X(\lambda^{\prime})$, ha $-\pi \le \lambda^{\prime} < \lambda^{\prime \prime} \le \pi$ 
					\end{itemize}
		\section{Becslések}
			\subsection{Várható érték becslése}
				\paragraph{}
					Legyen a várható érték becslése:
					$$\bar{X} = \cfrac{1}{T} \sum_{i = 1}^{T}X_i$$
					\begin{itemize}
						\item $T$ az összes idő megfigyelése
						\item $X$ a sztochasztikus folyamat.
					\end{itemize}
					Mint minden más statisztikai becsléstől ettől is elvárjuk a torzítatlanságot.
				\subsubsection{Torzítatlansága a várható érték}
					\paragraph{}
						$$E(\bar{X}) = \cfrac{1}{T} \sum_{i = 1}^{T}X_i = \cfrac{1}{T} \sum_{i=1}^T \mu_X = \mu_X$$
						Így látszódik, hogy torzítatlan.
			\subsection{Kovariancia függvény becslése}
				\paragraph{}
					A becsléshez használt összefüggések a következőek:
					$$
					\begin{aligned}
						\hat{R}_{1,k} = \cfrac{1}{T}\sum_{j=1}^{T -|k|}(X_j - \mu_X)(X_{j+|k|} -\mu_X) \\
						\bar{R}_{1,k} = \cfrac{1}{T-|k|} \sum_{j=1}^{T - |k|}(X_j - \mu_X)(X_{j+ |k|} - \mu_X) 
					\end{aligned}$$
					Ahol:
					\begin{itemize}
						\item $T$ hosszú folyamatunk van
						\item $k$ a "késeltetés" két megfigyelés között
						\item $\mu_X$ a várható érték, ha ez nem ismert érdemes becsülni.
					\end{itemize}
		\section{Fehérzaj folyamat}
			\paragraph{}
				Olyan stacionárius folyamat, amely
				\begin{itemize}
					\item korrelálatlan sorozatot alkot
					\item várható értéke 0
					\item Minden időpillanatban megegyezik az eloszlása.
				\end{itemize}
			\subsection{Fehérzaj folyamat spektrális sűrűségfüggvénye}
				$$g_\epsilon (\lambda) = \cfrac{1}{2* \pi} \sum_{j = - \infty}^{\infty} R_{\epsilon}(j)e^{-i j \lambda}$$
				ahol:
				\begin{itemize}
					\item $R_\epsilon(j)$ az eredeti fehérzaj folyamat autokorrelációs függvénye, amelyet az $\epsilon$ szűrővel szűrtek.
				\end{itemize}
			\subsection{standard fehérzaj folyamat}
				\paragraph{}
					Olyan fehérzaj folyamat, amelynek a varianciája minden időpontra pontosan 1.
			\subsubsection{standard fehérzaj folyamat spektrális sűrűségfüggvénye}
				$$g_\epsilon(\lambda) = \cfrac{1}{2 \pi} \sigma^2_\epsilon, \\\ -\pi \le \lambda \le \pi$$
		\section{Harmonikus folyamatok}
			\paragraph{}
				$$X_t = A_0 + \sum_{i=1}^q (A_i cos(\lambda_i t) + B_i sin(\lambda_i t))$$
				\begin{itemize}
					\item $X_t$-t nevezzük harmonikus folyamatnak, ha
					\item $A_i$ és $B_i$ korrelálatlan [[valószínűségi változó|valószínűségi változók]]
					\item $A_i$ és $B_i$ [[várható érték|várható értékei]] 0-k
					\item $VAR(A_0) = \sigma_0^2$
					\item $VAR(A_i)= VAR(B_i) = \sigma_i^2$
					\item $\lambda \in \mathbb{R}$, $q \in \mathbb{Z}^+$
				\end{itemize}
			\subsection{Harmonikus folyamatok tulajdonságai}
				\paragraph{}
					\begin{itemize}
						\item Periodikusak
						\item csak a két időpont közötti távolságtól függ a két időpont korrelációja
						\item Gauss eloszlásúak
						\item nem létezik spektrális sűrűségfüggvénye
					\end{itemize}
		\part{Lineáris folyamatok}
			\paragraph{}
				Elsőként szeretném szemléltetni a lineáris folyamatok használatát a következő példával:
			\paragraph{}
				Tegyük fel, hogy azt szeretnénk vizsgálni, hogy hogyan változnak az egyik tőzsdén jegyzett részvényáraink az idő függvényében. Az idősor analízis esetében a részvényárfolyamokat időpontonként mérjük és rögzítjük, és az időpontok közötti különbségek általában egyenlők.
			\paragraph{}
				Az ilyen típusú adatokat legegyszerűbb módon az autokorrelációs függvények segítségével modellezni. Az autokorrelációs függvény egy olyan matematikai eszköz, amely azt mutatja meg, hogy milyen erős az idősorbeli adatok közötti kapcsolat, azaz hogy az előző adatok milyen mértékben hatnak az időpontban mért adatokra.
			\paragraph{}
				Az autokorrelációs függvényekből pedig kiszámítható a spektrális sűrűségfüggvény is, amely az idősorbeli adatok frekvenciára vetített változását írja le. Ez azért fontos, mert az idősorbeli adatokban található frekvenciák meghatározzák az adatok jellemzőit, például azt, hogy milyen időtartamú ingadozások jellemzik az adott idősorbeli adatokat.
			\paragraph{}
				Ezeket a matematikai eszközöket alkalmazva az idősorbeli adatokat leíró lineáris folyamatokat lehet definiálni. Az ilyen lineáris folyamatok modelljei pedig felhasználhatók az idősorbeli adatok további vizsgálatára, például a trendek, szezonális változások vagy az idősorbeli adatok szabályszerűségeinek feltárására.
			\paragraph{}
				Egy ilyen példa egy lineáris folyamatra, amely az idősorbeli adatokat modellezi, lehet az ARIMA (AutoRegressive Integrated Moving Average) modell.
			\paragraph{}
				Azt a $X_t$ folyamatot nevezzük lineáris folyamatnak, amely teljesíti a következő:
				$$X_t = \sum_{s= -\infty}^{\infty} a_s \epsilon_{t-s}$$
				Ahol:
				\begin{itemize}
					\item $X_t$ egy sztochasztikus folyamat.
					\item $a_s \in \mathbb{R}$ és $\sum_{s = - \infty}^{\infty} a_s^2 < \infty$.
					\item $\forall \epsilon$ fehérzaj folyamat, $\sigma_\epsilon^2 > 0$.
					\item $\lim_{m,n \to \infty} E \left(X_t - \sum_{s = -m}^n a_s \epsilon_{t-s} \right)^2 = 0$.
				\end{itemize}

			\section{Kauzális folyamat}
				\paragraph{}
					A kauzális folyamat olyan lineáris folyamat, amely előállítható a következő képen:
					$$X_t = \sum_{s=0}^\infty a_s \epsilon_{t-s}$$
					Emellett az $X_t$ csak a múltjától függ. Ezen felül egy lineáris folyamat akkor és csak akkor kauzális folyamat, ha a spektrális sűrűségfüggvénye létezik és teljesül rá a Bochner-Kolmogorov tétel:
					$$\int_{-\pi}^\pi \ln(g_X(\lambda)) d\lambda > -\infty$$
					\begin{itemize}
						\item $g_X$ a spektrális sűrűségfüggvény
						\item $\lambda$ időbeli változások sebességét leíró paraméter
					\end{itemize}
					

				\subsection{Kauzális folyamat spektruma}
					\paragraph{}
					\begin{itemize}
						\item A kauzális folyamat spektruma kauszális: $g_X(\lambda) = 0$ minden $\lambda < 0$ esetén.
						\item $g_X(0) = \text{var}(X_t)$, azaz a spektrális sűrűségfüggvény értéke a zérus frekvencián megegyezik a folyamat varianciájával.
						\item A spektrális sűrűségfüggvény Fourier-transzformáltja pozitív mértékű véges mértékű eloszlás.
						\item Az autokovariancia függvény Fourier-transzformáltja az ún. spektrális eloszlásfüggvény, amelynek integrálja a folyamat varianciájával egyezik meg.
						\item Ha $X_t$ kauzális és $Y_t = f(X_t)$, ahol $f$ egy lineáris időinvariáns szűrő, akkor $Y_t$ is kauzális és $g_Y(\lambda) = |f(\lambda)|^2 g_X(\lambda)$.
						\item A kauzális folyamatokat jellemezhetjük az ún. kauzalitási impulzusfüggvényükkel, amely az $X_t$ folyamat egy egység-Dirac impulzus bemenetre adott válaszfüggvénye.
					\end{itemize}
			\section{Lineáris szűrő}
				\paragraph{}
					Lineáris szűrőt jelölje $L$.
					$$L(X_t) = \sum_{s = -\infty}^\infty h(t-s)X_s$$
					
					\begin{itemize}
					  \item Lineáris szűrőt jelölje $L$
					  \item $L(X_t) = \sum_{s = -\infty}^\infty h(t-s)X_s$
					  \item $X_t$ egy vagy több dimenziós stacionárius folyamat (tágabb értelemben)
					  \begin{itemize}
						    \item Van a folyamatnak $R_x(u)$-val definiált kovarianciafüggvénye
						    \item $g_X(\lambda)$ spektrális sűrűségfüggvény is létezik
						    \item $E(X_t) \equiv 0$
					  \end{itemize}
					  \item $h(t)$ függvény eleget tesz a szűrő koherenciafeltételnek: 
					  $$\sum_{u = -\infty} ^\infty \sum_{v = -\infty}^\infty h(u) R_X(v-u) h^T(v) < \infty$$
					\end{itemize}


				\subsection{transzfer függvény}
					\paragraph{}
						A szűrő koherenciafeltételhez tartozik az úgynevezett transzfer függvény:
						$$H \left(e^{-i \lambda} \right) = \sum_{t = -\infty}^{\infty} e^{-i t \lambda} h(t)$$
						\begin{itemize}
						    \item $H$ a transzfer függvény
						    \item $\lambda$ az időbeli változások sebességét leíró paraméter.
						\end{itemize}
					\paragraph{}
						Legyen $Y_t = L(X_t)$ stacionárius folyamat, ahol:
						\begin{itemize}
						    \item $L$ egy lineáris szűrő
						    \item $X_t$ egy stacionárius folyamat
						\end{itemize}
						
						Ekkor az $Y_t$ kovarianciafüggvényére fennáll:
						$$R_Y(r,r+t) = R_Y(t)$$
						Ahol:
						\begin{itemize}
						    \item $r$ a vizsgált időpont
						    \item $t$ pedig, hogy mennyi időegységre van eltolva a vizsgálat vége az $r$-től
						\end{itemize}
					\paragraph{}
						Az $Y_t$ spektrális sűrűségfüggvényére pedig:
						\begin{itemize}
						    \item Több dimenziós esetben a $g_Y(\lambda) = H\left(e^{-i \lambda} \right)g_X(\lambda)H^{T*}\left(e^{-i \lambda} \right)$ összefüggés igaz.
						    \item Egy dimenziós esetben pedig a $g_Y(\lambda) = g_X(\lambda)\left|H\left(e^{-i \lambda} \right)\right|^2$ összefüggés igaz.
						\end{itemize}
						Ahol:
						\begin{itemize}
						    \item $g$ a spektrális sűrűségfüggvény
						    \item $\square^{T*}$ a transzponált, majd konjugált műveletek jelölik.
						\end{itemize}
					\paragraph{}
						$H \left(e^{-i \lambda} \right)$ esetén a kauzális folyamat spektrális sűrűségfüggvénye a
						$$g_X(\lambda) = \cfrac{\sigma_\epsilon^2}{2 \pi} \left| \sum_{t=0}^\infty a_t e^{-it\lambda}\right|^2$$ lesz.
			\section{autókorreláció függvény}
				\paragraph{}
					Megmutatja, hogy a folyamat időbeli részei mennyire hasonlítanak egymáshoz, mennyire van közös mozgásuk.
					Legyen $X_t$ stacionáriu] folyamat $R_X(t)$ pedig kovarianciafüggvény, ekkor az autokorrelációs függvényt a következő képen értelmezzük:
					$$r_X(t) = \dfrac{1}{R_X(0)}R_X(t) = \cfrac{1}{\sigma_X^2}R_X(t)$$.
				\subsection{Autókorreláció függvény spektruma}
					\paragraph{}
						A következőt nevezzük az autokorrelációs függvény spektrumának:
						$$g_X(\lambda) = \cfrac{1}{2 \pi} \sum_{k = - \infty} ^{\infty} r_X(k) e^{-i * k *\lambda}$$
						A zaj, az interferencia vagy a periodikus mintázatok kezelésében. A spektrum azonban csak egy részleges képet ad a folyamatról, és egyedül nem elegendő a folyamat jellemzéséhez. Számos más jellemzőnek, mint például az autoregressziós modell paramétereinek, a szűrő paramétereinek vagy a spektrális sűrűségfüggvénynek ismerete szükséges lehet a folyamatok részletes elemzéséhez.
				\subsection{parciális autokorreláció}
					\paragraph{}
						$Z_1, \dots, Z_k$ valószínűségi változók melletti parciális korrelációnak nevezzük a következő összefüggést:
						$$\rho = \mathrm{CORR}(X - \hat{X}, Y - \hat{Y}) = \frac{\mathrm{COV}(X-\hat{X}, Y - \hat{Y})}{D(X - \hat{X})D(Y - \hat{Y})}$$
						Ahol:
						\begin{itemize}
							\item $X,Y,Z$ valószínűségi változók
							\item $\hat{X} = \hat{X}(Z_1, \dots, Z_k), \hat{Y} = \hat{Y}(Z_1, \dots, Z_k)$
							\item $D(X - \hat{X}) > 0, D(Y - \hat{Y}) > 0$.
						\end{itemize}
						Megmutatja az $X$ és $Y$ közötti kapcsolat erősségét azután, hogy mindkét változóban kiküszöböltük a $Z$ valószínűségi változók hatását.
					\subsubsection{parciális autokorreláció stacionárius folyamat esetén}
						\paragraph{}
							Azt mutatja meg, hogy két időbeli pontra vonatkozó korreláció mennyiben magyarázható a közöttük lévő időbeli pontok hatásának kiküszöbölésével. Stacionárius folyamatok esetén a parciális autokorreláció különösen fontos, mivel lehetővé teszi számunkra, hogy azonosítsuk a folyamat autoregresszív (AR) modellének paramétereit.
							$X_t$ és $X_{t-k}$ közötti parciális autokorreláció alatt a következő összefüggést értjük:
							\[
							\rho_k = CORR(X_t - \widehat{X}_t, X_{t-k} - \widehat{X}_{t}) = \frac{COV(X_t - \widehat{X}_t, X_{t-k} - \widehat{X}_{t})}{D(X_t - \widehat{X}_t)D(X_{t-k} - \widehat{X}_{t})}
							\]
							ahol:
							\begin{itemize}
								\item $k \in \mathbb{Z}$,
								\item $\widehat{X}_t = \widehat{X}_t(X_{t-1}, \dots, X_{t-k+1})$,
								\item $D(X_t - \hat{X}_t) > 0$ és $D(X_{t-k} - \hat{X}_t) > 0$.
							\end{itemize}
			
			\section{AR; MA; ARMA folyamatok}
				\paragraph{}
					Ide valami arról, hogy mire való
				\subsection{AR}
					\paragraph{}
						A p-edrendű autoregresszív modellnek a következőt hívjuk:
						$$x_t = \sum_{k=1}^{p}(c_k x_{t-k}) + \epsilon_t$$
						\begin{itemize}
							\item $x_t$ a $t.$ időpontban lévő véletlen stacionárius folyamat értéke
							\item $c_i$ az $i.$ időponthoz tartozó autoregressziós együttható
							\item $p$ az autoregressziós modell rendje
							\item $\epsilon_t$ a $t.$ időponthoz tartozó diszkrét fehérzaj.
						\end{itemize}
					\subsubsection{autoregressziós együttható}
						\paragraph{}
							Az autoregressziós együttható az autoregresszív modellben a korábbi időpontok értékeinek súlyozására használt együttható.
				\subsection{MA - Mozgóátlag modell}
					\paragraph{}
						EZ NAGYON NEM TETSZIK
						A mozgóátlag szűrőnek a következőt szokás hívni:
						$$x_t = d(B) \epsilon_t$$
						Ahol:
						\begin{itemize}
							\item $d(B) = \sum_{i=0}^\infty d_iB^i$
							\begin{itemize}
								\item $B^n(x_t) = x_{t-n}$
							\end{itemize}
							\item $\epsilon_t$ pedig [[Fehérzaj folyamat|fehérzaj]]
							\item $x_t=\sum_{k = 1}^{q}(d_{k}\epsilon_{t-k})+\epsilon_t$
						\end{itemize}
					\subsubsection{Mozgóátlag együttható}
						A mozgóátlag együtthatók az előző hibák lineáris kombinációját jelzik, amelyek súlyozásával a jelenlegi értéket becsülik.
				\subsection{ARMA}
					\paragraph{}
						Az ARMA folyamat előállítása az mozgóátlag modell (MA) és az Autoregresszív modell (AR) segítségével történik:
						$$AR|MA(p,q) = \sum_{k = 1}^p (c_k x_{t-k}) + \sum_{k=1}^q(d_k \epsilon_{t-k}) + \epsilon_t$$
						\begin{itemize}
							\item $p,q \in \mathbb{Z}^+$
							\begin{itemize}
								\item $p$ az Autoregresszív modell (AR) rendje
								\item $q$ a mozgóátlag modell (MA) rendje
							\end{itemize}
							\item $c_k$ a $k.$ időponthoz tartozó autoregressziós együttható
							\item $x_k$ a $k.$ időponthoz tartozó tágabb értelemben stacionárius folyamat értéke
							\item $d_k$ a $k.$ időponthoz tartozó mozgóátlag együttható
							\item $\epsilon_t$ fehérzaj
							\begin{itemize}
								\item várható értéke 0
								\item szórása nagyobb, mint 1
							\end{itemize}
						\end{itemize}
				\subsection{karakterisztikus polinom}
					\paragraph{}
						Egyáltalán nem értem, csak annyit, hogy van a Laplace traszformálás: $F(s) = \int_0^\infty e^{-st}f(t)dt$ és ebből jönnek ki a $z$ változók a polinomhoz, ami segít előállítani a $P(z) = 1-\sum_{k=1}^{p} a_k z^k$. 
					\paragraph{}
						Ha ennek a gyökei az egységkörön belül helyezkednek el, akkor ez egy kauzális folyamat.
				\subsection{Yule-Walker egyenletek}
					\paragraph{}
						A Yule-Walker egyenletrendszer az autoregresszív modell paramétereinek becslésére használt lineáris egyenletekből álló rendszer, ami a következő formában van:
						
						$$\gamma = R_p * c$$
						
						\begin{itemize}
						\item $\gamma$ az autokorrelációs függvény a $k.$ értékeit tartalmazza
							$$\gamma = \left[ \begin{array}{c}
							\gamma_1 \\ \gamma_2 \\ \vdots \\ \gamma_p
							\end{array}\right]$$
						\item $R_p$ szintén az autokorrelációs függvény $k.$ értékeit tartalmazza egy kicsit másmilyen formátumban.
							$$R_p = \left[\begin{array}{cccccc}
							\gamma(0) & \gamma(1) & \gamma(2) & \dots & \gamma(p-1) \\
							\gamma(1) & \gamma(0) & \gamma(1) & \dots & \gamma(p-2) \\
							\gamma(2) & \gamma(1) & \gamma(0) & \dots & \gamma(p-3) \\
							\vdots & \vdots & \vdots & \ddots & \vdots \\
							\gamma(p-1) & \gamma(p-2) & \gamma(p-3) & \dots & \gamma(0)
							\end{array}\right]$$
						\begin{itemize}
							\item $\gamma(0)$ általában $1$
						\end{itemize}
						\item $c$ az autoregressziós együtthatók, ezeket szeretnénk becsülni.
							$$c = \left[\begin{array}{c}
							c_1 \\ c_2 \\ \vdots \\ c_p
							\end{array}\right]$$
						\end{itemize}
				\subsection{Paraméterek becslése}
					\paragraph{}
						Az ARMA modellek paraméterbecslési módszerei közé tartoznak:
						\begin{itemize}
							\item \textbf{AIC (Akaike információs kritérium)}: Ez a módszer arra törekszik, hogy egy egyszerű, de hatékony modellt találjon, amely a legjobban illeszkedik az adathalmazhoz. Az AIC egy információs mutató, amely a modell illeszkedését és a modell bonyolultságát egyaránt figyelembe veszi. Az AIC értéke alapján két vagy több modell közül választhatunk, ahol az alacsonyabb érték jobb illeszkedést jelent.
							\item \textbf{BIC (Bayes információs kritérium)}: Ez a módszer hasonló az AIC-hez, de a modell bonyolultságát még jobban figyelembe veszi. A BIC értéke alapján választhatunk a különböző modellek közül, ahol az alacsonyabb érték jelzi a jobb illeszkedést.
							
							\item \textbf{Maximum likelihood (maximum valószínűség)}: Ez a módszer az ARMA modell paramétereinek becslésére szolgál. A maximum likelihood módszer az a feltevés, hogy az adathalmaz az ARMA modellből származik, és az ARMA modell paramétereinek becslése olyan értékeket keres, amelyek a lehető legjobban magyarázzák az adatokat.
							
							\item \textbf{Mínimális négyzetek módszere}: Ez a módszer az AR modell paramétereinek becslésére szolgál. A módszer az AR modell paramétereit úgy becsüli meg, hogy minimalizálja a modell és az adatok közötti négyzetes eltérést. Ez a módszer egyszerű és könnyen alkalmazható, de csak az AR modellekhez alkalmazható.

					\end{itemize}
		\part{Wiener folyamatok}
			\paragraph{}
				Akkor nevezünk egy folyamatot Wiener folyamatnak ($W(t), t\ge 0$), ha következő tulajdonságokat teljesítik:
				\begin{itemize}
					\item Független növekményű Gauss-folyamat
					\item Trajektóriái 1 valószínűséggel folytonosak
					\item $W(0) = 0$
					\item Várható értéke 0 minden $t$-re
					\item $VAR(W(t)) = \sigma^2 * t$
					\item $t$ időváltozó, $t \ge 0$
				\end{itemize}.
			\section{Wiener folyamat tulajdonságai}
				\paragraph{}
					\begin{itemize}
						  \item $COV(W(t), W(s)) = \min(s,t)$
						  \begin{itemize}
						   	\item $t,s \ge 0$ ezek egymástól független időpontokat jelölnek a Wiener folyamat útján
						  \end{itemize}
						  \item $VAR(W(t)-W(s))=|t-s|$
						  \begin{itemize}
						  	\item $t,s \ge 0$ ezek egymástól független időpontokat jelölnek a Wiener folyamat útján
						  \end{itemize}
						  \item $\overline{W}(t) = -W(t)$ is Wiener-folyamat
						  \begin{itemize}
						  	\item $t \ge 0$
						  \end{itemize}
						  \item $\overline{W}(t) = W(t+s_0) - W(s_0)$ is Wiener-folyamat, amely nem függ $W_s, 0 \le s \le s_0$ folyamattól
						  \begin{itemize}
						  	\item $t \ge 0$
						  \end{itemize}
						  \item $cW(t/c^2)$ Wiener folyamat
						  \begin{itemize}
						  	\item $c > 0$ valamilyen konstans
						  	\item $t \ge 0$ időváltozó
						  	\item automodalitásnak nevezzük
						  \end{itemize}
						  \item Rendelkezik Markov tulajdonsággal
						  \begin{itemize}
						 	  \item aktuális állapota azonnal meghatározza annak jövőbeni állapotát
						  	 \item nincs memóriája az előző állapotokról
						  \end{itemize}
						\item $\lim_{t \to \infty} \dfrac{W_t}{t}=0$ nagy számok erős törvénye szerint, 1 valószínűséggel
						\item Sehol sem differenciálható
					\end{itemize}.

			\section{Standard Wiener folyamat}
				\paragraph{}
					Amennyiben a Wiener folyamat teljesíti, hogy $\sigma = 1$ akkor azt standard Wiener folyamatnak nevezzük.
			\subsection{Standard Wiener folyamat plusz tulajdonságai}
				\paragraph{}
					\begin{itemize}
						\item Standard Wiener folyamat véges dimenziós eloszlásainak sűrűségfüggvénye: 
						$$\begin{aligned}
							f(y_1, \dots, y_m; t_1, \dots , t_n) &= (2\pi)^{-n/2}\left(t_1*\prod_{k=2}^n(t_n-t_{n-1}) \right)^{1/2} \cdot \\
							&\cdot e^{-\cfrac{1}{2} \left( \cfrac{y_1^2}{t_1}+\sum_{k=2}^n \cfrac{(y_n-y_{n-1})^2}{t_n - t_{n-1}} \right)}
						\end{aligned}$$
						\begin{itemize}
							\item $0 < t_1 < t_2 < \dots < t_n \infty$
							\item $y_1, \dots, y_n \in \mathbb{R}$ 
						\end{itemize}
						\item iterált logaritmus tételből következik, hogy:
						\begin{itemize}
							\item $\limsup_{t \to \infty} \dfrac{W_t}{\sqrt[2]{2*t*\ln(\ln(t))}}=1$
							\item $\liminf_{t \to \infty} \dfrac{W_t}{\sqrt[2]{2*t*\ln(\ln(t))}}=-1$
							\item ezek leírják a Wiener folyamat aszimptotikus viselkedését
							\item létezik olyan $t_0$ időpont, amelytől igaz, hogy: $-(1+ \epsilon)*\sqrt[2]{2*t*\ln(\ln(t))} \le W_t \le (1+ \epsilon)*\sqrt[2]{2*\ln(\ln(t))}$
							\begin{itemize}
								\item $\epsilon > 0$
							\end{itemize}
						\end{itemize}
					\end{itemize}
			\section[Wiener-folyamat konstrukciója]{Konstrukciója}
				\paragraph{}
					A Wiener folyamat konstrukciójának általános alakja:
					$$W(t) = \sum_{k=0}^{\infty}X_k \int_0^t \phi_k (u) du$$
					\begin{itemize}
						\item $t$ az idő változó
						\begin{itemize}
							\item $t$-t elegendő $[0,1]$ intervallumon megadni, mert független növekményű Gauss folyamatról van szó
						\end{itemize}
						\item $X_k \sim N(0,1)$
						\item $\phi$ ortonormált bázis a $\mathcal{L}^2[0,1]$ L2 függvénytéren
					\end{itemize}
				\subsection{Wiener-féle konstrukció}
					\paragraph{}
						Egy másik konstrukciója a Wiener folyamatnak a Wiener-féle konstrukció:
						$$W(t) = \cfrac{t}{\sqrt{\pi}} X_0 + \sqrt{2} \sum_{n=1}^\infty \sum_{k = 2^{n-1}}^{2^n-1}\sqrt{\cfrac{2}{\pi}}\cfrac{sin(k*t)}{k} X_k$$
						\begin{itemize}
							\item $t$ időváltozó
							\item $X_k$ a $k.$ időpontban lévő sztochasztikus folyamat
						\end{itemize}
				\subsection{Lévy-Ciesielski-féle konstrukció}
					\paragraph{}
						$$W(t) = S_0(t)X_0 + \sum_{n=1}^{\infty} \sum_{k \text{ (páratlan) }=1 }^{2^{n}}S_{k2^{-n}}(t)X_{k2^{-n}}$$
						\begin{itemize}
							  \item Ahol $S_k(t)$-t Schauder-függvények:
							  $$S_k(t) = \int_{0}^t h_k(u)du$$
							    \begin{itemize}
							      \item És $h_k$ pedig Haar-függvényt jelent:
							      $$\begin{aligned}
							         h_0(x) & \equiv 1 \\
							         h_{k 2^{-n}}(x) & = \begin{cases}
							             +2^{(n-1)/2}, & (k-1)2^{-n} \le x < k2^{-n}, \\
							             -2^{(n-1)/2}, & k2^{-n} \le x < (k+1)2^{-n}, \\
							             0 & \text{egyébként}
							         \end{cases}
							       \end{aligned}$$
							    \end{itemize}
							  \item $X \sim N(0,1)$
						\end{itemize}	
			\section[Wiener-folyamat trajektóriáinak viselkedése]{Trajektóriák viselkedése}
				\paragraph{}
					$$
					\lim_{n \to \infty} \text{VAR} \left( \sum_{k=1}^{n} \left[ W\left(t_k^{(n)}\right) - W\left(t_{k-1}^{(n)}\right) \right]^2 - (t-s) \right) = 0
					$$
					\begin{itemize}
						\item $W(t)$: a standard Wiener folyamat.
						\item $t_k^{(n)}$: az időintervallumok, $t$ és $s$ közötti felosztása $n$ egyenlő részre.
					\end{itemize}
				\paragraph{}
					$$\lim_{N \to \infty} \sum_{k=1}^{2^N}\left| W \left( \cfrac{k}{2^N}\right) - W \left( \cfrac{k-1}{2^N}\right) \right| = \infty$$
					\begin{itemize}
						\item $W(t)$: a standard Wiener folyamat.
						\item $N$: a szintek száma, amelyekre az időintervallumok fel vannak osztva ($N \to \infty$ tartalmazza a határesetet).
						\item $\frac{k}{2^N}$: az időintervallumok kezdőpontjai, amelyek az időtartományt $2^N$ egyenlő részre osztják.
					\end{itemize}
				\paragraph{}
					Wiener folyamat 1 valószínűséggel sehol sem differenciálható. Más szóval, a Wiener folyamat trajektóriái az időben folyamatosan változnak, ugrásokkal és "fordulókkal", és nincsenek első deriváltjaik egyetlen pontban sem. Ez azt jelenti, hogy a Wiener folyamat trajektóriái nem simák, és nem lehet őket differenciálással leírni egyetlen pontban sem.
				\paragraph{}
					$$P\left( \max_{0 \le s \le t} W(s) \ge x\right) = 2P(W(t) \ge x) = 2 \left( 1- \Phi\left(\left( \cfrac{x}{\sqrt{t}} \right)\right) \right) $$
					\begin{itemize}
						\item $t$: Időpont, amely az időtartamot jelöli, ahol az $x$ értéket meghaladja a Wiener folyamat maximuma. Ez egy valós szám, $t \geq 0$.
						\item $x$: Egy küszöbérték, amely felett megvizsgáljuk a Wiener folyamat maximumának eloszlását az időpontban $t$. Ez egy valós szám, $x \in \mathbb{R}$.
						\item $W(t)$: Wiener folyamat értéke az időpontban $t$. Ez egy sztochasztikus folyamat, amelyet a Brown-mozgásként is ismerünk, és a normális eloszlású véletlen változók szummájaként definiáljuk. $W(t)$ egy valós szám.
						\item $\Phi(z)$: Standard normális eloszlás eloszlásfüggvénye. Ez egy sztochasztikus változótól független függvény, amely az $z$ valós számot képezi le a $[0,1]$ intervallumra. Az $z$ változó az $x$ értéket meghaladó Wiener folyamatot adja vissza az időpontban $t$ normált eloszlásban.
					\end{itemize}
				\paragraph{}
					Egy valószínűséggel fennáll az iterált logaritmus tétel a Wiener folyamatokra:
					$$\lim \sup_{t \to \infty} \cfrac{W(t)}{\sqrt{2t \log(\log(t))}} = 1$$
					$$\lim \inf_{t \to \infty} \cfrac{W(t)}{\sqrt{2t \log(\log(t))}} = -1$$
			\section{Ito-féle sztochasztikus integrál}
				\paragraph{}
					$$\lim_{n \to \infty}\sum_{k=1}^{n} f(z_{k-1})[W(z_k)- W(z_{k-1})] = \int_{a}^b f(s) dW(s)$$
					\begin{itemize}
						\item Eredménye: $0$ várható értékű normális eloszlású valószínűségi változó.
						\item $\Delta W(z_k) = W(z_{k+1}) - W(z_k) \sim N(0, \Delta t)$
						\item $f(z_{k-1})$ egy ismert szám, ezt szorozzuk a $\Delta W(z_k)$-val.
						\item Rendelkezik a Stieltjes integrál tulajdonságaival.
					\end{itemize}
				\subsection{Ito-féle sztochasztikus integrál tulajdságai}
					\paragraph{}
						\begin{itemize}
							  \item Az $f$ függvény $dW$ szerinti integráljának várható értéke $0$.
							  \item Az integrál négyzetének várható értéke megegyezik az $f$ függvény négyzetének Riemann integráljával.
							  \item $W(t)$-nek egy pontja sem differenciálható.
						\end{itemize}
			\section{Ito Lemma}
				\paragraph{}
					Minden integrálra minden rögzített $t \ge t_0$ esetén fennállnak a következő tulajdonságok:
				\subsection{integrál linearitása}
					\paragraph{}
					$$\int_{t_0}^t (a*G_1 + b*G_2) dW =\int_{t_0}^t a*G_1 dW + \int_{t_0}^t b*G_2 dW$$
					ahol:
					\begin{itemize}
						\item $a,b \in \mathbb{R}$
						\item $G_1, G_2 \in M_2[t_0,_t]$ lépcsős függvénye
					\end{itemize}
				\subsection{Vektorintegrál}
					\paragraph{}
						$$\int_{t_0}^t G dW = \left(\begin{array}{c}
							\sum_{k=1}^m \int_{t_0}^t G_{ik}(s)dW_s^k \\
							\vdots \\
							\sum_{k=1}^m \int_{t_0}^t G_{dk}(s)dW_s^k
						\end{array} \right);
						W_t = \left(\begin{array}{c}
							W_t^1 \\
							\vdots \\
							W_t^m
						\end{array}\right)$$
				\subsection{integrál átlaga nulla tulajdonság}
					\paragraph{}
						Ha $E(|G(s)|) < \infty \forall t_0 \le s \le t$, akkor
						$$E\left( \int_{t_0}^t G dW \right)=0$$
				\subsection{Ito-Kunita formula második pontja}
					\paragraph{}
						$E(|G(s)|)^2 < \infty \forall t_0 \le s \le t$ akkor a $d \times d$ kovarianciamátrixára fennáll
						$$E\left( \int_{t_0}^t G dW * \left( \int_{t_0}^t G dW \right)^\prime \right) = \int_{t_0}^t E(G(s)*G(s)^\prime)ds$$
						és
						$$E\left( \left| \int_{t_0}^t G dW \right|^2 \right) = \int_{t_0}^t E(|G|^2)ds$$
			\section{Wiener-Paley lemma}
				\paragraph{}
					Minden $G \in M_2[t_0, t]$ függvényhez létezik $M_2[t_0,t]$-beli lépcsős függvények egy $G_m$ sorozata, úgy, hogy érvényes a következő összefüggés.
					$$P\left[ \lim_{n \to \infty} {\int_{t_0}^t {|G(s) - G_n(s)|^2}ds } = 0\right] = 1$$
			\section{Doob általánosított egyenlőtlensége}
				\paragraph{}
					Ha $G \in M_2[t_0,t]$ egy lépcsős függvény, akkor a következő érvényes:
					$$
					P \left[ \left| \int_{t_0}^tG(s)dW_s \right| > c \right] \le \dfrac{N}{c^2} + P \left[ \int_{t_0}^t |G(s)|^2 ds > N \right]
					$$
					ahol: $N>0$ és $c > 0$.
			\section{Doob-Meyer tétel}
				\paragraph{}
					$${st.\lim_{n \to \infty}} \int_{t_0}^t | G(s) - G_n(s)|^2ds = 0$$
					ahol:
					\begin{itemize}
					\item ${st. \lim}$ azt jelenti, hogy sztochasztikus középben vett határértéke
					\item $G, G_n \in M_2[t_0,t]$ lépcsős függvények.
					\end{itemize}
					Amennyiben igaz, hogy $\int_{t_0}^t G_n(s) dW_s$ által van definiálva az érték, akkor igaz az is, hogy
					$${st. \lim_{n \to \infty}} \int_{t_0}^t G_n(s) dW_s = I(G)$$
					ahol: $I(G)$ a $\{G_n\}$ sorozat megválasztásától független valószínűségi változó.
			\section{Sztochasztikus integrálás}
				\paragraph{}
					$$\begin{aligned}
						\int_{t_0}^t G(s,\omega) dW_s(\omega)
					\end{aligned}$$
					ahol:
						\begin{itemize}
						\item $t_0, t$ az intervallum amin értelmezzük az integrálást
						\item $G \in M_2^{d,m}[t_0,t]$
						\item $W_s(\omega)$ kifejezés egy adott $\omega$ minta esetén az $s$ időpillanatban mért standard Wiener-folyamat értékét jelöli.
						\item $s$ az időpillanat, amiben vizsgálódunk.
					\end{itemize}.
					\begin{itemize}
						\item Megvizsgáljuk, hogy $G$ lépcsős függvény-e, tehát létezik-e olyan felbontás, amelyre igaz az, hogy $t_0 < t_1 < \dots < t_n = t$ és $G(s) = G(t_{t-1}) \forall s \in [t_{i-1},t_i[, i=1,\dots,n$, ennek tulajdonságai az Ito lemmában vannak leírva.
						\item Megkeressük a $G_m$ sorozatot, amely eleget tesz a Wiener-Paley lemmának.
						\item Majd minden függvényt átalakítunk úgy, hogy eleget tegyen a Doob általánosított egyenlőtlenségének.
						\item Ellenőrizzük, hogy eleget tesz-e a Doob-Meyer tételnek.
						\item Elvégezzük rajta az Ito-féle sztochasztikus integrálást.
					\end{itemize}
			\section{Sztochasztikus differenciál}
				\paragraph{}
					$$X_t(\omega) = X_{t_s}(\omega) + \int_{t_0}^t f(s, \omega)ds + \int_{t_0}^t G(s, \omega) dW_s(\omega)$$
					
					Ahol:
					\begin{itemize}
						\item $W_t$ egy $m$ dimenziós Wiener-folyamat
						\item $G \in M_2^{d,m}[t_0, T]$
						\item $X_{t_s}$ sztochasztikus folyamat, amely független a $W_t - W_{t_0}$ zajtól, $t \ge t_0$
						\item $f$ egy függvény, amely független a $W_t - W_{t_0}$ zajtól, $t \ge t_0$, és $R^d$ értékkészletű. A jövőtől nem függő, és $1$ valószínűséggel teljesül, hogy $\int_{t_0}^T |f(s, \omega)|ds < \infty$.
						\item $\omega$ a mintatér eleme, azaz egy adott kimenet az eseményteret alkotó $\sigma$-algebra egy eleme
						\item $s$ a függvény értékelésének az időpillanata.
						\item $d$ a véges differenciálás lépése.
						\item $m$ az a Wiener-folyamat dimenziója. Ha egy Wiener-folyamat $m$ dimenziós, akkor azt jelenti, hogy az $m$ darab független standard Wiener-folyamatot összefűztük egy $m$ dimenziós vektorban.
					\end{itemize}
					\subsection{Sztochasztikus differenciál jelölése és értelmezése}
						\paragraph{}
							$$dX_t = f(t)dt + G(t) dW_t = fdt + G dW$$
							$$X_t - X_s = \int_s^t f(u)du + \int_s^t G(u)dW_u$$
					\subsection{Ito lemma sztochasztikus differenciálegyenletekre}
						\paragraph{}
							$$
							\begin{aligned}
								\dfrac{\delta}{\delta t}u(t,x)&=u_t& \\
								\dfrac{\delta}{\delta x_i}u(t,x)&= u_{x_i} & x = (x_1, \dots, x_d) \\
								\dfrac{\delta^2}{\delta x_i \delta x_j} u(t,x) &= u_{x_i x_j} & i,j\le d
							\end{aligned}
							$$
							Ahol:
							\begin{itemize}
								\item $u:(t,x) \to \mathbb{R}^k$
								\begin{itemize}
									\item folytonos
									\item minden $k$ dimenziós vektorértékű parciális deriváltjai is folytonosak
									\end{itemize}
									\item $t \in [t_0, T]$
									\item $x \in \mathbb{R}^d$.
							\end{itemize}
						\paragraph{}
							Legyen a következő sztochasztikus differenciál egyenlet:
							$$d X_t = f(t)dt + G(t) d W_t$$
							ahol:
							\begin{itemize}
								\item $X_t$ az $d$ dimenziós sztochasztikus folyamat $[t_0, T]$-ben
								\item $W_t$ pedig $m$ dimenziós Wiener-folyamat.
							\end{itemize}
						\paragraph{}
							Akkor tudjuk értelmezni a
							$$Y_t = u(t, X_t)$$
							$k$ dimenziós folyamatot. Amelynek a kezdeti értéke $Y_{t_0} =u(t_0,X_{t_0})$.
							$$
							\begin{aligned}
								Y_t = u(t, X_t) \\
								Y_{t_0} =u(t_0,X_{t_0}) \\
								d Y_t = & \bigg[ u_t(t, X_t) + u_x(t,X_t) * f(t) + \\
								&+ \dfrac{1}{2} * \sum_{i=1}^d \sum_{j=1}^d u_{x_i, x_j}(t,X_t)  (*G(t)*G(t)^\prime_{ij})
								\bigg]dt + \\
								&+u_x(t,X_t)*G(t) dW_t
							\end{aligned}
							$$.
						\paragraph{}
							Ezt írhatjuk egy kicsit egyszerűbb formában:
							$$
							\begin{aligned}
								d Y_t = & \bigg[ u_t(t, X_t) + u_x(t,X_t) \cdot f(t) + \\
								&+ \dfrac{1}{2} \cdot tr(G \cdot G^\prime \cdot u_{xx})
								\bigg]dt + \\
								&+u_x(t,X_t) \cdot G(t) dW_t \\
								tr =& \sum_{i=1}^d \sum_{j=1}^d u_{x_i, x_j}(t,X_t)  \cdot (G(t) \cdot G(t)^\prime)_{ij}
							\end{aligned}
							$$
							Ahol:
							\begin{itemize}
								\item $u_x = (u_{x1}, \dots, u_{xk})$ $k \times d$ méretű mátrix
								\item $u_{x_i x_j}$ pedig $k$ dimenziós oszlopvektor
								\item $u_{xx} = (u_{x_i x_j})$ $d \times d$ méretű mátrix, amelynek elemei $k$ dimenziós vektorok.
							\end{itemize}
\end{document}